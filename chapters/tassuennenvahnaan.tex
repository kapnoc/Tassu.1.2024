
\section{Tassu ennen vanhaan}
\textit{Tassu on ilmestynyt säännöllisen epäsäännöllisesti lippukunnan perustamisvuodesta 1986
lähtien. Tällä palstalla muistellaan menneitä ja julkaistaan valittuja paloja takavuosien
lehdistä.}

\textit{Päätoimittaja sai paketin, jossa on vuosikymmeniä Tassun painoksia ja
yrittää digitalisoida ne lähiaikoina!}

\vspace*{0.32cm}
\noindent Tällä kertaa mennään ajassa 20 vuotta taaksepäin, ja katsotaan mitä
\mbox{KuRu:n} kolkat puuhasivat heidän syysretkellään:

\newtcolorbox{StickyNote}[1][]{%
    enhanced,
    before skip=2mm,after skip=2mm, 
    width=\textwidth, boxrule=0.2mm, % width of the sticky note
    colback=kuru!50!white, colframe=kuru, % Colors
    attach boxed title to top right={xshift=0cm,yshift*=0mm-\tcboxedtitleheight},
    varwidth boxed title*=-3cm,
    % The titlebox:
    boxed title style={frame code={%
        \path[left color=kuru,right color=kuru,
        middle color=kuru]
        ([xshift=-0mm]frame.north west) -- ([xshift=0mm]frame.north east)
        [rounded corners=0mm]-- ([xshift=0mm,yshift=0mm]frame.north east)
        -- (frame.south east) -- (frame.south west)
        -- ([xshift=0mm,yshift=0mm]frame.north west)
        [sharp corners]-- cycle;
        },interior engine=empty,
    },
    sharp corners,rounded corners=southeast,arc is angular,arc=3mm,
    % The "folded paper" in the bottom right corner:
    underlay={%
        \path[fill=kuru!80!black] ([yshift=3mm]interior.south east)--++(-0.4,-0.1)--++(0.1,-0.2);
        \path[draw=kuru,shorten <=-0.05mm,shorten >=-0.05mm,color=kuru] ([yshift=3mm]interior.south east)--++(-0.4,-0.1)--++(0.1,-0.2);
        },
    % drop fuzzy shadow, % Shadow
    % fonttitle=\bfseries, 
    title={#1}
}

\vspace*{0.64cm}
\begin{StickyNote}[Julkaistu alunperin Tassussa 1/2004]
	\monofont
	% \white

	\textbf{\Large Kolkkaretkellä kuultua}
	\vspace*{0.64cm}

	Kolkat retkeilivät 1.-2.11. Kuusituvalla Vantaalla. Alla on otteita retken ohjelmasta:

	\vspace*{0.64cm}
	\fbox{\begin{minipage}{\linewidth}
		- Mikä täällä haisee? Kamala haju! Mitä sä teet Kirsku?\\
		- Keitän teille päivälliseksi käpysoppaa, mausteena on vähän
		sammalta. Nämä on ne erhut jotka te aamupäivällä keräsitte.\\
		- YÄK!! Mä en ainakaan tommosta syö. Et sä voi pakottaa meitä
		syömään tota.\\
		- Ei vaineskaa. Tää soppa on tarkoitettu solmunarujen ja
		villalangan värjäämiseen, ei syötäväksi. Vaikka kyllä se
		siihenkin sopii, sehän olisi pihkan makuista teetä. Anssi jo
		koemaistoi sitä.
	\end{minipage}}

	\vspace*{0.64cm}
	Värjäämisen ja Paavon ja Maijun kehittämän hämähäkkiaskartelun jälkeen kaikui
	komento: puukot kuistin pöydälle odottamaan, taskulamput ja
	muistiinpanovälineet mukaan ja ulos

	\vspace*{0.32cm}
	{\Large\ldots}
	\vspace*{0.32cm}
\end{StickyNote}

\begin{StickyNote}[Julkaistu alunperin Tassussa 1/2004]
	\monofont
	% \white

	{\Large\ldots}

	\vspace*{0.64cm}
	\fbox{\begin{minipage}{\linewidth}
		- Ottakaa parit. Metsään on Eddie ja Oskari merkanneet reitin
		heijastimilla. Maasto on melko helppoa, mutta kulkekaa
		varovasti. Matkalla on kysymyksiä. Yhdistäkää parit esim.
		\mbox{kauha + kattila}, \mbox{Citroen + Picasso}, ja
		kirjoittakaa vastaukset paperille.\\
		- Helppo reitti, hauskat kysymykset. Ai oliko siellä
		kysymyksiä? pitikö niihin vastata?\\
		- Palauttakaa vastaukset ja hakekaa puukot. Etsikää Anssi ja
		toimikaa hänen ohjeidensa mukaan.\\
		- Makkaran paistoa!! Nam, nam!
		% \ldots
	\end{minipage}}
	\vspace*{0.32cm}

	Virkistävän yöunen jälkeen nousimme reippaina. Satoi vettä, tietysti!
	Onneksi oli ohjelmaa sisällä.

	\vspace*{0.32cm}
	\fbox{\begin{minipage}{\linewidth}
		- Uudet kolkat tänne, vanhat tuon pöydän ääreen! Nyt
		harjoitellaan solmuja. Ensin se tärkein, merimiessolmu, vanhat
		voivat harjoitella jalussolmua.\\
		- Onks tää oikein? No entäs tää?\\
		- Toi ei ole oikein, yritä uudelleen! Tää on oikein, hyvä!
	\end{minipage}}
	\vspace*{0.32cm}

	Solmujen jälkeen oli jälleen aika vähän kilpailla.

	\vspace*{0.32cm}
	\fbox{\begin{minipage}{\linewidth}
		- Ottakaa hämähäkkinne esiin ja ottakaa pari. Laittakaa
		hämähäkki keskelle pöytää ja alkakaa puhaltaa.\\
		- Puuh! Puuh! Puuh!! Mä voitin!!
	\end{minipage}}
	\vspace*{0.32cm}

	Vielä ennen kotiinlähtöä etsimme vihjerasteja ja siivosimme. Toki teimme
	retkellämme myös paljon muuta: leikimme ulkona ja sisällä, haravoimme ajotien,
	söimme yms.

	\vspace*{0.32cm}
	Kolkista mukana olivat: Pottu, Dalmis, Ruusu, Hai, Joutsen, Hamsteri, Ahven,
	Metso ja Skorppioni sekä harjoittelijana Pinja

	\vspace*{0.64cm}
	Retkimuistoja kokosi: Kirsku
\end{StickyNote}
