\section{Lippukunnanjohtajan tervehdys}

\begin{multicols}{2}
\noindent Jälleen yksi toiminnantäyteinen partiovuosi on vierähtänyt ja pian on aika rauhoittavan joululoman ennen seuraavan vuoden seikkailuja. Myös edellisestä Tassusta on ehtinyt vierähtää jo useampi vuosi. Erityiskiitos uuden päätoimittajan rusakoiden oma lippukuntalehti on saanut tuulta purjeisiinsa ja me kaikki pystymme palaamaan kuluneen vuoden tapahtumiin erilaisten juttujen ja kuvien merkeissä -- jo 75. kertaa!

Ehkä ensimmäistä kertaa koskaan Tassusta löytyy ei yksi, ei kaksi, vaan peräti kolme raporttia vuoden aikana suoritetuista nahkaliljavaelluksista (s.~\pageref{section:vihreaNahkalilja}, \pageref{section:mustaNahkalilja} ja \pageref{section:punainenNahkalilja}). Nahkaliljat ovat partiomerkkejä, joiden tarkoituksena on tähdentää ulkoilun merkitystä ja innostaa partiolaisia ylläpitämään ja kehittämään peruskuntoaan. Ruumiillisesti ja henkisesti kuormittavat vaellukset ovat vartiossa toimimista parhaimmillaan, tapahtuvat luonnossa ja muuttuvat nousujohteisesti yhä haastavammiksi taitettavan matkan pidentyessä. Juuri päivitetyssä partiomenetelmässä käytetään ilmaisuja ryhmässä toimiminen, toiminta luonnossa ja oma partiopolku -- voihan vaelluksia kutsua myös elämyksiksi.

Ei uutta ilman vanhaa, myös partioliikkeen perustaja Baden"-Powell korostaa terveyden ja liikunnan tärkeyttä viimeisessä viestissään partiolaisille: ''Ensi askel onneen on tämä: kasvata itsesi terveeksi ja voimakkaaksi, niin että vartuttuasi voit olla hyödyksi ja siten nauttia elämästä.'' Terveyteen kuuluu muitakin ulottuvuuksia kuin fyysinen terveys ja partion tavoitteena on kasvattaa lapsista ja nuorista persoonallisuudeltaan ja elämäntavoiltaan tasapainoisia yhteiskunnan jäseniä.

Partiomenetelmää sovelletaan kaikkeen toimintaan partiossa: viikkokokouksiin, erilaisiin retkiin, paraateihin, kilpailuihin, leireihin, vaelluksiin, juhliin ja muihin tapahtumiin. Tassu voi olla paljon muutakin kuin sanallista ja kuvallista kuvausta toiminnasta. On tärkeää, että lippukuntalehti on lippukuntansa näköinen. Mikäli haluaisit osallistua lehden tekemiseen, vedä rohkeasti päätoimittajan hihasta. Osallistu ainakin Tassun kuvakilpailuun. Löydät ohjeet sivulta~\pageref{section:kuvakilpailu}!

Partioterveisin

Janne

\medskip

\noindent\includegraphics[width=\linewidth,trim={0 1.5cm 0 4cm},clip]{assets/lpkjtervehdys}

\medskip

\noindent\null\hfill Kuva: Tanguy Gérôme
\end{multicols}
