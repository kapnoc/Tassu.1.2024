\section{Musta nahkalilja 4.--5.10. KESKEN}

\begin{multicols}{2}
\noindent Seuraan Helsingin päärautatieaseman laiturilla neljä, kuinka junayksiköitä yhdistetään pidemmäksi, Riihimäelle suuntaavaksi junaksi. Nousen junaan. Seikkailu voi alkaa. Tavoitteenani on kävellä sata kilometriä kahdessakymmenessäneljässä tunnissa -- viimeinen, tosin epävirallinen nahkalilja, musta nahkalilja. Toukokuun 2015 harmaasta nahkaliljastani oli ehtinyt jo vierähtää tovi ja nyt jännittikin, kuinka kävely maistuu liki kymmenen vuotta myöhemmin.

Junassa ei ole kuulutuksia eivätkä opasnäytöt ole päällä. Etenkin Keravan jälkeen alkaa epäilyttää, osaako tässä jäädä oikealla asemalla pois, kun ikkunasta näkyy vain sysimustaa eivätkä laituirien asemanimet ole aina helposti luettavissa. ''Hyvinkää'' erottuu kuitenkin selkeästi ja hyppään pois junasta. 

Juna saapuu aikataulunsa mukaisesti Hyvinkäälle kello 19.51. Siirryn asemarakennuksen eteen odottamaan lähtölaukaisua. Olin laatinut itselleni oman aikatauluni, jota seuraisin vaellukseni aikana. Vaellus oli jaettu kahteenkymmeneen likimain viiden kilometrin etappiin ja jokaiselle etapille olin laskenut odotetun ohitusajan kolmella eri keskinopeudella: Suurimmalla mahdollisella nopeudella, jolla uskoin pystyväni kävelemään koko matkan (4,9~km/t), nopeudella, jolla matka taittuisi kahdessakymmenessäkolmessa tunnissa jättäen yhden tunnin joustoa odottamattomien tapahtumien varalle (4,35~km/t), ja nopeudella, jolla juuri ja juuri ehtisin maaliin kahdessakymmenessäneljässä tunnissa (4,17~km/t).

\begin{table*}[t]
\small\centering
\begin{tabular}{|c|ccc|c|}
\hline
\multirow{3}{*}{\textbf{Matka (km)}} & \multicolumn{3}{c|}{\textbf{Nopeus (km/t)}} & \multirow{3}{*}{\textbf{Paikka}} \\ \cline{2-4}
 & \multicolumn{1}{c|}{4.90} & \multicolumn{1}{c|}{4.35} & 4.17 &  \\ \cline{2-4}
 & \multicolumn{3}{c|}{\textbf{Kellonaika}} &  \\ \hline
0 & \multicolumn{1}{c|}{19:56:00} & \multicolumn{1}{c|}{19:56:00} & 19:56:00 & Hyvinkää \\ \hline
5 & \multicolumn{1}{c|}{20:57:13} & \multicolumn{1}{c|}{21:05:00} & 21:07:57 & Palorinteent. \\ \hline
10 & \multicolumn{1}{c|}{21:58:26} & \multicolumn{1}{c|}{22:14:00} & 22:19:54 & Haapasaarent. \\ \hline
15 & \multicolumn{1}{c|}{22:59:40} & \multicolumn{1}{c|}{23:23:00} & 23:31:51 & Mastot. \\ \hline
20 & \multicolumn{1}{c|}{00:00:53} & \multicolumn{1}{c|}{00:32:00} & 00:43:48 & Nuppulinnank. \\ \hline
25 & \multicolumn{1}{c|}{01:02:07} & \multicolumn{1}{c|}{01:41:00} & 01:55:45 & Vanhankylän kt. \\ \hline
30 & \multicolumn{1}{c|}{02:03:20} & \multicolumn{1}{c|}{02:50:00} & 03:07:42 & Holjanrinne \\ \hline
35 & \multicolumn{1}{c|}{03:04:34} & \multicolumn{1}{c|}{03:59:00} & 04:19:39 & Sarvikalliont. \\ \hline
40 & \multicolumn{1}{c|}{04:05:47} & \multicolumn{1}{c|}{05:08:00} & 05:31:36 & Kulloont. \\ \hline
45 & \multicolumn{1}{c|}{05:07:01} & \multicolumn{1}{c|}{06:17:00} & 06:43:33 & Keravant. \\ \hline
50 & \multicolumn{1}{c|}{06:08:14} & \multicolumn{1}{c|}{07:26:00} & 07:55:30 & Sipoont. \\ \hline
55 & \multicolumn{1}{c|}{07:09:28} & \multicolumn{1}{c|}{08:35:00} & 09:07:27 & Pohjolant. \\ \hline
60 & \multicolumn{1}{c|}{08:10:41} & \multicolumn{1}{c|}{09:44:00} & 10:19:24 & Kehä III \\ \hline
65 & \multicolumn{1}{c|}{09:11:55} & \multicolumn{1}{c|}{10:53:00} & 11:31:21 & Mikaelinkirkko \\ \hline
70 & \multicolumn{1}{c|}{10:13:08} & \multicolumn{1}{c|}{12:02:00} & 12:43:18 & Lähdenväylä \\ \hline
75 & \multicolumn{1}{c|}{11:14:22} & \multicolumn{1}{c|}{13:11:00} & 13:55:15 & Tapaninkylänt. \\ \hline
80 & \multicolumn{1}{c|}{12:15:35} & \multicolumn{1}{c|}{14:20:00} & 15:07:12 & Lautamiehenp. \\ \hline
85 & \multicolumn{1}{c|}{13:16:48} & \multicolumn{1}{c|}{15:29:00} & 16:19:09 & Silvolant. \\ \hline
90 & \multicolumn{1}{c|}{14:18:02} & \multicolumn{1}{c|}{16:38:00} & 17:31:06 & Kehä I \\ \hline
95 & \multicolumn{1}{c|}{15:19:15} & \multicolumn{1}{c|}{17:47:00} & 18:43:03 & Pekka Jokipaltion t. \\ \hline
100 & \multicolumn{1}{c|}{16:20:29} & \multicolumn{1}{c|}{18:56:00} & 19:55:00 & Siilitie \\ \hline
\end{tabular}
\caption{Mustan nahkaliljan etapit ja suunniteltu ''aikataulu''.}
\end{table*}

Lähtölaukaisu tulee \textbf{kello 19.56} ja starttaan kohti Jokelankatua. Hyvinkäällä on täysi perjantaitohina päällä ja varsin paljon ihmisiä on liikkeellä. Ensimmäinen etappi on helppo, oleellisesti vain suoraan Jokelankadun viereistä kevyenliikenteenväylää pitkin. Ohitseni ajaa mopoilijoita ja nuori pyöräilijäpari Jopollaan, toinen satulassa ja toinen tavaratelineellä. Tavaratelineellä istuva tähtää punaisella takavalolla taakseen olkansa yli. Valtatie 25:n tietämillä mopoilijat ovat pysähtyneenä Jokelantien varteen, kun ilmeisimmin toisessa mopossa on käynnistymisvaikeuksia. Samalla katuvalaistus Jokelantien varresta päättyy.

Vastaan ajavat autot sokaisevat voimakkailla valoillaan ja joudun säätämään takkini hupun toimimaan eräänlaisena vastavalosuojana. Taivas on kirkas ja tähdet erottuvat huomattavasti selkeämmin kuin Helsingissä. Harmi vain kuu ei valaissut maisemaa (toim. huom. kuu laski jo kello 18.30) ja mustalla asfaltilla pysyminen on ajoittain haastavaa. Saavutan ensimmäisen etapin \textbf{kello 20.49} -- viisi kilometriä takana ja yhdeksänkymmentäviisi edessä. Samalla kuin kirjaan saapumisaikaa aikatauluuni, yksi mopoilijoista juoksee ajoradan varressa moponsa vieressä selvästi yrittäen saada sitä käynnistymään, taluttaa mopon hengästyneenä kevyenliikenteenväylälle ja tervehtii lyhyesti: ''Moi.''

Reitti kaartaa aavistuksen itään pientä lisäkierrosta varten ja kääntyy Haapasaarentielle. Alitan Pääradan. Samankaltainen kierros suunniteltiin Tuusulaan, jotta reitistä Mikaelinkirkolle, jossa mustan nahkaliljan reitti yhtyy vihreään ja punaiseen, saataisiin kuusikymmentäviisi kilometriä pitkä. Näin mustan nahkaliljan maali osui kolmensadan metrin päähän kodistani ja nopea huolto suorituksen jälkeen oli taattu.

Katuvalot vieläkin uupuvat, miksi kaivan esiin otsalampun tiellä pysymistä helpottaakseni. Autoliikennettä on vähemmän, mikä on positiivista, koska tiellä ei ole jalkakäytävää. Vastaani ajaa traktori, joka monine lisävaloineen näyttää etäältä aivan Transformers"-autobotilta. Tilojen koirat alkavat haukkua aina, kun kävelee niiden ohi, ja haukunta jatkuu pitkään ohituksen jälkeen. Parhaimmillaan taisin kuulla, kun kuusi koiraa haukkui samaan aikaan yön pimeyteen. Toisen etapin saavutan jo \textbf{kello 21.41}, yli seitsemäntoista minuuttia parhaintakin aikataulua edellä. Alun innostustani olen kävellyt tauotta varsin nopeasti, keskimäärin 5,7~km/t, joten pidän keksinmittaisen tauon Katilan ja Riihelän tilojen välissä.

Alkumatkan reitillä ei ole kovin montaa käännöstä, mikä helpottaa pimeässä suunnistamista suunnattomasti. Uusikyläntiellä on jalkakäytävä ja pian myös katuvalot. Autoja ajaa ohi harvakseltaan. Muita ulkoilijoita ei näy. Kävely on kevyttä vaikkakin ilma alkaa tuntua vähän viileältä. Repussa on vielä yksi ohut villapaita, mutta säästän sitä myöhemmäksi. Tien nimi vaihtuu Ridasjärventieksi; kolmas etappi osuu kahden kunnan rajalle \textbf{kello 22.35}: Toisen kunnan vaakunassa on kolme syöstävää (toim. huom. tekstiiliteollisuuteen liittyvä sukkula), toisen pistoolinlukko ja laakerinoksa. Syön välipalapatukan ja karjalanpiirakan. Olen varannut mukaan evästä liki kymmenen tuhannen kilokalorin edestä ja yritän popsia jotain aina puolen tunnin välein.

Höyhensaarentie ja Lepolan linja"-autopysäkki muistuttavat aikoja sitten menneestä nukkumaanmenoajasta. Olo on vielä virkeä, mutta se taitaa johtua yksin rautatieasemalla nautitusta kahvista. Kaksi koiranulkoiluttajaa kummastelee menoani Tehtaantieltä. Alitan taas Pääradan ja reitti palaa Jokelantielle, jota pitkin kävelen Jokelan keskustan läpi. Ilma peltojen yläpuolella on aavemainen, kun koko näkymä on ohuen sumun peitossa. Paikoin sumu on sankenpaa muodostaen yhtenäisiä muotoja ilmaan. Tie on suora ja eteneminen vauhdikasta. Jäniksenlinna"-tieviitta tuo mieleen monet kesäiset pyöräretket, joita olen seudulla tehnyt. 

Neljännen etapin saavutan \textbf{kello 23.43}, vieläkin oikein reilusti aikataulua edellä. Tällä kertaa en pysähdykään etappipisteessä tauolle, vaan jatkan hetken eteenpäin toiveenani löytää mielekäs istumapaikka. Sellaista ei kuitenkaan tule heti vastaan, joten pysähdyn Nuppulinnan linja"-autopysäkille lyhyelle evästauolle ja pukemaan viimeisen villapaidan päälleni. Kaikki mukana olevat vaatteet on nyt päällä.

Ymmärrän olevani niin paljon aikataulua edellä, että päätän pitää vielä toisen tauon ennen seuraavaa etappia. Sopiva istumapaikka löytyy Ankkapuron linja"-autopysäkiltä, jossa pysähdyn hetkeksi lepuuttamaan jalkojani. Pysäkki on niin huonossa kunnossa, että koko pysäkkikatos kallistuu painostani, kun istun sen penkille -- aivan kuin pysäkki ei olisi kiinni maassa lainkaan. Tauon ajan pysäkki kuitenkin kestää ja yhden Nutella"-leivän jälkeen pysäkki jää pystyyn, kun jatkan matkaani.

Neljäsosa liljasta on takana, kun saavun viidennelle etapille \textbf{kello 0.44}. Kylmä ja kostea ilma alkaa puskea vaatekerrosten läpi ja alan katua sitä, että jätin kotiin toiset välihousut ja toppaliivin. Kylmän vuoksi vauhtia pitää nopeana ja taukoja lyhyinä. Pian etapin jälkeen saavun taas kunnan rajalle: Jokelantie vaihtuu Eriksnäsintieksi. Tällä kertaa kunnan vaakunassa on suuri siivekäs lyyra. 

Nenääni leijailee voimakas tanniininen, punaviinimainen haju. Ihmettelen suuresti, menestyisikö näin sivussa asutuksesta mikään juomaravintola. Samalla vastaani juoksee yölenkkeilijä. Neljän sadan metrin jälkeen hajun lähde selviää. Tien varressa on Kiertokapula"-nimisen jätteidenkäsittelyalueen pilari, jonka kello näytää tasan kello 1.00 ja lämpötila kolme astetta. 

Tie päättyy Vanhankyläntielle, joka on ennestään tuttu kymmeniltä pyörälenkeiltä. Suureksi yllätyksekseni Tuusulanjärveä kiertävä kevyenliikenteen väylä on poikki välillä Ratsukatu--Röynänkatu vesihuollon saneeraustöiden vuoksi. En uskalla lähteä kiertoreitille pimeään tuntemattomaan, joten poikkelehdin työmaan ohitse. Kunta vaihtuu hetkeksi takaisin pistoolinlukkoon ja laakerinoksaan vain palatakseen siivekkääseen lyyraan. Saavun kuudennelle etapille \textbf{kello 1.40}. Fiilis on hyvä ja harmitusta aiheuttaa vain kylmä keli. 

Reitti jatkuu järven ympäri kiertäen palaten pistoolinlukon ja laakerinoksan kuntaan. Kävely tuntuu siltä kuin automaattiohjaus olisi päällä. Aikani kuluksi mietiskelen teiden kaltevuuksia ja erityisesti sitä, koituuko siitä myöhemmin jotain harmia, jos kävelee koko ajan alustalla, joka on kolme prosenttia kallellaan samaan suuntaan. Siispä vaihdan aina hetkeksi kevyenliikenteenväylän oikeasta laidasta vasempaan, kun en parempaa tekemistä keksi.

Seitsemäs etappi tulee vastaan Sarvikallion tienoilla \textbf{kello 2.44}. Uni maittaisi oikein hyvin, mutta se saa odottaa, koska edessä on vielä kuusikymmentäviisi kilometriä. En pidä taukoa vaan jatkan kävelyä. Tie kaartaa etelään ja pidän sukkienvaihtotauon Paijalannummen linja"-autopysäkillä. Sukkia vaihtaessa syön välipalapatukkaa, josta tietenkin valahtaa muru väärän torven puolelle. Yskäkohtauksen jälkeen saan kaiken kukkuraksi vielä piinallisen hikan. Kuinka helposti sitä voikin tulla huonolle tuulelle!

Hikottelen eteenpäin ja yritän kaikkia tuntemiani konsteja hikan karkoittamiseksi. Reilun kilometrin päästä tulen Tuusulanjoen padolle. Joen ylittävän sillan pinta on jäätynyt ja olen liukastua sitä kävellessäni. Liukastuminen säikäyttää minut niin, että hikka kaikkoaa. Jotain hyvää lienee pahassakin. (Toim. huom. Tuusulanjoki laskee Vantaanjokeen, joka ylitetään reitillä ensimmäisen kerran noin kahdeksankymmenenkuuden kilometrin kohdalla.)

Reitti kulkee Tuusulan lävitse. Teen pienen poikkeaman reitiltä Hyrylässä Koskenmäentien ylityksessä, koska ylityspaikkaa ei kohdassa ole ja on kuljettava alikulusta. Saavun kahdeksannelle etapille Kulloontien varteen \textbf{kello 3.47} ja jälleen jätän tauon välistä. 

Seuraava osuus on mielestäni kaikista tylsin. Kunta vaihtuu hirren sinkkausliitokseen, alitan taas Pääradan ja ylitän Keravanjoen. (Toim. huom. Päärata alitetaan reitillä vielä kerran noin seitsemänkymmenenkolmen ja ylitetään kerran noin yhdeksänkymmenenneljän kilometrin kohdalla. Lisäksi Keravanjoki ylitetään reitillä myös toistamiseen noin seitsemänkymmenenseitsemän kilometrin kohdalla. Myös Keravanjoki laskee Vantaanjokeen.) Saavun Lahdentielle etapille yhdeksän \textbf{kello 4.43} -- ja Lahdentietä sitten jatketaankin noin kolmetoista kilometriä. 

En ehdi kovin montaa metriä Lahdentietä kävellä, kun \textit{Bää} pyöräilee minua vastaan. Sovitusti hän tuo minulle eväs- ja vesitäydennyksen sekä aamukahvin -- kermaa unohtamatta. Istun juomaan kahvia Jokelan tilaa vastapäätä oleville puurappusille, jotka johtavat Isotuvan pelloille. Samalla huomaan ulompien reisilihasteni alkaneen kipeytyä ja yritän niitä hieroa ja venytellä. 

Tavanomaista pidemmän tauon jälkeen toivotan \textit{Bäälle} hyvää kotimatkaa ja jatkan kävelyä. Pidempi tauko ja kylmä ilma ovat ehtineet jäykistää jalkojeni lihakset ja liikkeelle lähteminen, ''lentokorkeuteen pääseminen'' kestää huomattavan pitkään. Viimein pääsen tuttuun kävelyrytmiini eikä kävelyä itseään tarvitse enää niin tietoisesti ajatella. 

Kunta vaihtuu kalanpyrstöön ja pääsen nahkaliljan puoliväliin: etappi kymmenen, \textbf{kello 6.03}. Enää viisikymmentä kilometriä edessä. Myös muita ulkoilijoita, koiranulkoiluttajia ja pyöräilijöitä alkaa näkyä ja valon määrä kasvaa. Kunta vaihtuu muutamaan otteeseen kalanpyrstön ja sudenpään välillä. ''Ei ylimääräisiä taukoja,'' ajattelen ja pusken eteenpäin.

Etapille yksitoista saapuessani \textbf{kello 7.03} reiteni ovat oikeasti kipeät, joten pysähdyn niitä aktiivisesti venyttelemään ja syömään eväitä. Tauon aikana ihan yllättäen olo muuttuu hyvin väsyneeksi ja uupuneeksi. Koen heikkoa pahoinvointia enkä tahdo löytää maasta tai seisten asentoa, jossa johonkin ei särkisi. Lahdentietä Helsinkiin päin ajaa linja"-auto seitsemänsataakolmekymmentäyksi ja koen keskeyttämisen halun erityisen suureksi. Tällaisina hetkinä pari tai ryhmä olisivat ensiarvoisen tärkeitä, kun oma unenpuutteinen mieli ja fyysiesti rasittunut keho tekevät tenän. Pitkät fyysiset suoritukset muuttuvat päänsisäisiksi kamppailuiksi.

Lukuisten minuuttien jälkeen saan vakuuteltua itseni taas liikkeelle. Lahdentien päähän päästyäni huokaisen helpotuksesta ja Kehä III tuntuu kuin kodilta. Etappi kaksitoista osuu Kehä III:n ylittävälle sillalle \textbf{kello 8.00}, josta kipuan Kuussillan mäelle taukoa pitämään ja kuvainnollisesti ''miettimään elämänvalintojani''. 

Länsimäentiellä reitti yhtyy viime kevään nahkaliljaan ja kunta vaihtuu veneeksi. Teen toisen poikkeaman suunnitellulta reitiltä ja päätän alittaa Kontulantien liikennevaloissa seisomisen sijaan. Saavun Mikaelinkirkolle, etapille kolmetoista hyvissä ajoin, \textbf{kello 9.01}. Kirkko ei ole auki, mutta pääsen taukoilemaan Mikon luokse ja sitten nauttimaan aamuauringosta Päiväkoti Kontulan pihalle, jossa vaihdan sukat. Tästä reitti jatkuu takaisin pohjoiseen yhdessä vihreän ja punaisen nahkaliljan suorittajien kanssa -- lue lisää heidän raportistaan!

{\smallskip\noindent\centering ***\par\smallskip}

Erkanen osastosta \textit{ViPuMu} Viikissä lähellä Mäyrämetsää viimeisellä etapilla \textbf{kello 19.17}, jotta ehdin varmasti maaliin Siilitielle vaaditussa ajassa. Kulkeminen on hyvin kankeaa ja vasta yli puolen kilometrin jälkeen pystyn askeltamaan enemmän tai vähemmän normaalisti. Mäki Herttoniemen urheilupuistoon on armoton, mutta onnistun sen huiputtamaan ja saavun viimeiselle etapille kaksikymmentä, mustan nahkaliljan maaliin \textbf{kello 19.35}. Olo on väsynyt mutta voitonriemuinen. Onneksi koti on lähellä.

Koko matkan pituudeksi tuli tarkistusmittauksen jälkeen noin satakaksi kilometriä, mikä tarkoitti noin 4,3 kilometrin tuntivauhtia -- varsin haipakkaa siis.

Ilmatieteen laitoksen Hyvinkäänkylän havaintoasemalla merkittiin vaelluslauantain vastaisen yön sääksi selkeää, lämpötila 0--6 astetta ja ilmankosteus 92--98 prosenttia. Kumpulan havaintoasemalla merkittiin vaelluslauantaipäivän sääksi selkeää, lämpötila 5--14 astetta, länsituulta 1--3 metriä sekunnissa ja ilmankosteus 49--94 prosenttia -- mitä erinomaisin nahkaliljakeli, sanoisin!
\end{multicols}

\vspace*{.50cm}
{\raggedleft Kuva: NN\\
Teksti: Janne Suomalainen\par}
