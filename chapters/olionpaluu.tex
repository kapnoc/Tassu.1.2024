
\clearpage
\subsection{Olion paluu 19.--21.4. KESKEN}

\begin{multicols}{2}

	\noindent Jatkona viime syksyn \textit{Operaatio olio} "=telttaretkelle lippukunnan
	seikkailijavartio \textit{Makkaramokkulat} ja tarpojavartio \textit{Päärynähyttyset} lähtivät
	telttailemaan samoihin Sipoonkorven maastoihin retken jännittävässä jatko"-
	osassa \textit{Olion paluu}, jonka \textit{Vene}"-vartio oli jälleen menestyksekkäästi
	suunnitellut.

	Kaksitoista rusakkoa kokoontui retken lähtöön jo tutussa paikassa, Kontulan
	ostarin Kalkan Pizza"-Kebab"-Grillin edustalla perjantai"-iltapäivänä. Kun osa
	retkeläisistä hyppäsi bussiin kohti Viikkiä, osa siirtyi lippukunnan
	varastolle, jossa logistiikkamestari Mikko odotti valmiina hakemaan retken
	ruoat ja kyyditsemään ne kaluston -- ja kantajien -- kanssa telttapaikalle.
	Bussiosasto jatkoi Viikistä Sipoon Myyrakseen.

	Bussiosastolla oli edessä tuskien taival: 2,4 kilometrin kävely pysäkiltä
	telttapaikan läheiselle parkkipaikalle. Toisaalta, ei auto"-osastollakaan
	helppoa ollut kantaa kaikki retken ruoat ja yhteiset varusteet parkkipaikalta
	ylös leiriin. Vaikka matkaa oli vain 700 metriä, oli leiripaikka viisitoista
	metriä korkeammalla merenpinnasta kuin parkkipaikka. Tehtävästä suoriuduttiin
	kuitenkin ansiokkaasti.

	Telttapaikalla aloitettiin leirityöt: osa retkeläisistä alkoi pilkkoa puita
	yötä varten kun taas osa alkoi pystyttää telttaa. Teltanpystytys ei sujunut
	aivan määräajassa, mutta lopputulos oli särmä ja erittäin maastokelpoinen.
	Selvästi telttaa ei käytetä tarpeeksi, kun kulmasalkojen asentaminen on niin
	paljon voimaa vaativa työ.

	Tässä vaiheessa retkeä alkoi satamaan lunta ja lumentulo jatkuikin aina
	sunnuntaihin asti. Huhtikuista takatalvea ei jääty sen enempää ihmettelemään
	vaan retkeläiset siirtyivät teltan suojaan, jossa kamina syttyi melkein
	itsestään etukäteen valmisteltujen sytykkeiden avulla.

	Ennen nukkumaanmenoa sovittiin vielä kipinävuoroista: Ensimmäisinä
	kipinävuorossa valvoivat Elias ja Mikael, sitten Alden ja Toivo, Joella, Ninni
	ja Tesla, Ahti ja Kata, Johannes ja Touko, sekä viimeisenä Janne. ''Pakastaja
	Elvi'' ei päässyt yllättämään, vaikka kaminaa poltettiinkin säästöliekillä.
	Ainakin yhden kerran yön aikana ''lohikäärme'' tuli kuitenkin tarpeeseen, kun
	kyteville puille piti antaa vähän lisäilmaa. Niin ikään osalla retkeläisistä
	oli aamuksi jännitystä, kun Ahti pisti pystyyn F1"-kisastudion.

	Retkeläiset heräsivät valkeaan lauantaiaamuun: lunta oli tullut yön aikana
	useampi sentti ja lämpötila oli muutaman asteen pakkasen puolella. Väsyneet ja
	kohmeiset retkeläiset siirtyivät aamupuuron pariin. Ravittuina leikittiin
	sitten Katan johdolla kaikkien suosikkileikkejä kuten pölkkyä ja tervapataa --
	jälkimmäistä tosin vahvasti kontaktiversiona.

	\ldots

	\raggedleft Toimittaja: Janne Suomalainen\\

\end{multicols}

